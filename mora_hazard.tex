% Options for packages loaded elsewhere
\PassOptionsToPackage{unicode}{hyperref}
\PassOptionsToPackage{hyphens}{url}
\PassOptionsToPackage{dvipsnames,svgnames,x11names}{xcolor}
%
\documentclass[
  letterpaper,
  DIV=11,
  numbers=noendperiod]{scrartcl}

\usepackage{amsmath,amssymb}
\usepackage{iftex}
\ifPDFTeX
  \usepackage[T1]{fontenc}
  \usepackage[utf8]{inputenc}
  \usepackage{textcomp} % provide euro and other symbols
\else % if luatex or xetex
  \usepackage{unicode-math}
  \defaultfontfeatures{Scale=MatchLowercase}
  \defaultfontfeatures[\rmfamily]{Ligatures=TeX,Scale=1}
\fi
\usepackage{lmodern}
\ifPDFTeX\else  
    % xetex/luatex font selection
\fi
% Use upquote if available, for straight quotes in verbatim environments
\IfFileExists{upquote.sty}{\usepackage{upquote}}{}
\IfFileExists{microtype.sty}{% use microtype if available
  \usepackage[]{microtype}
  \UseMicrotypeSet[protrusion]{basicmath} % disable protrusion for tt fonts
}{}
\makeatletter
\@ifundefined{KOMAClassName}{% if non-KOMA class
  \IfFileExists{parskip.sty}{%
    \usepackage{parskip}
  }{% else
    \setlength{\parindent}{0pt}
    \setlength{\parskip}{6pt plus 2pt minus 1pt}}
}{% if KOMA class
  \KOMAoptions{parskip=half}}
\makeatother
\usepackage{xcolor}
\setlength{\emergencystretch}{3em} % prevent overfull lines
\setcounter{secnumdepth}{-\maxdimen} % remove section numbering
% Make \paragraph and \subparagraph free-standing
\ifx\paragraph\undefined\else
  \let\oldparagraph\paragraph
  \renewcommand{\paragraph}[1]{\oldparagraph{#1}\mbox{}}
\fi
\ifx\subparagraph\undefined\else
  \let\oldsubparagraph\subparagraph
  \renewcommand{\subparagraph}[1]{\oldsubparagraph{#1}\mbox{}}
\fi


\providecommand{\tightlist}{%
  \setlength{\itemsep}{0pt}\setlength{\parskip}{0pt}}\usepackage{longtable,booktabs,array}
\usepackage{calc} % for calculating minipage widths
% Correct order of tables after \paragraph or \subparagraph
\usepackage{etoolbox}
\makeatletter
\patchcmd\longtable{\par}{\if@noskipsec\mbox{}\fi\par}{}{}
\makeatother
% Allow footnotes in longtable head/foot
\IfFileExists{footnotehyper.sty}{\usepackage{footnotehyper}}{\usepackage{footnote}}
\makesavenoteenv{longtable}
\usepackage{graphicx}
\makeatletter
\def\maxwidth{\ifdim\Gin@nat@width>\linewidth\linewidth\else\Gin@nat@width\fi}
\def\maxheight{\ifdim\Gin@nat@height>\textheight\textheight\else\Gin@nat@height\fi}
\makeatother
% Scale images if necessary, so that they will not overflow the page
% margins by default, and it is still possible to overwrite the defaults
% using explicit options in \includegraphics[width, height, ...]{}
\setkeys{Gin}{width=\maxwidth,height=\maxheight,keepaspectratio}
% Set default figure placement to htbp
\makeatletter
\def\fps@figure{htbp}
\makeatother

\KOMAoption{captions}{tableheading}
\makeatletter
\@ifpackageloaded{caption}{}{\usepackage{caption}}
\AtBeginDocument{%
\ifdefined\contentsname
  \renewcommand*\contentsname{Table of contents}
\else
  \newcommand\contentsname{Table of contents}
\fi
\ifdefined\listfigurename
  \renewcommand*\listfigurename{List of Figures}
\else
  \newcommand\listfigurename{List of Figures}
\fi
\ifdefined\listtablename
  \renewcommand*\listtablename{List of Tables}
\else
  \newcommand\listtablename{List of Tables}
\fi
\ifdefined\figurename
  \renewcommand*\figurename{Figure}
\else
  \newcommand\figurename{Figure}
\fi
\ifdefined\tablename
  \renewcommand*\tablename{Table}
\else
  \newcommand\tablename{Table}
\fi
}
\@ifpackageloaded{float}{}{\usepackage{float}}
\floatstyle{ruled}
\@ifundefined{c@chapter}{\newfloat{codelisting}{h}{lop}}{\newfloat{codelisting}{h}{lop}[chapter]}
\floatname{codelisting}{Listing}
\newcommand*\listoflistings{\listof{codelisting}{List of Listings}}
\makeatother
\makeatletter
\makeatother
\makeatletter
\@ifpackageloaded{caption}{}{\usepackage{caption}}
\@ifpackageloaded{subcaption}{}{\usepackage{subcaption}}
\makeatother
\ifLuaTeX
  \usepackage{selnolig}  % disable illegal ligatures
\fi
\usepackage{bookmark}

\IfFileExists{xurl.sty}{\usepackage{xurl}}{} % add URL line breaks if available
\urlstyle{same} % disable monospaced font for URLs
\hypersetup{
  pdftitle={Moral Hazard},
  pdfauthor={Gonzalo Islas Rojas, Universidad Adolfo Ibáñez},
  colorlinks=true,
  linkcolor={blue},
  filecolor={Maroon},
  citecolor={Blue},
  urlcolor={Blue},
  pdfcreator={LaTeX via pandoc}}

\title{Moral Hazard}
\author{Gonzalo Islas Rojas, Universidad Adolfo Ibáñez}
\date{2015-01-01}

\begin{document}
\maketitle

\section{1. Introduction}\label{introduction}

In general terms, moral hazard arises when the value of a transaction
for one party can be influenced by actions or decisions taken by the
other party. Two additional conditions are required for the problem of
moral hazard to occur:

\begin{itemize}
\tightlist
\item
  Hidden action: This action is not directly observable or perfectly
  inferable by the party whose results are affected.
\item
  Risk aversion: At least one of the parties must be risk-averse.
\end{itemize}

The essence of the moral hazard problem lies in the conflict between two
fundamental objectives:

\begin{itemize}
\tightlist
\item
  Provision of adequate incentives
\item
  Efficient risk distribution
\end{itemize}

There is a substitution effect: the extent to which one party is
shielded from risk influences their incentives to take actions that are
efficient for the other party.

The challenge is to design an incentive mechanism that minimizes cost,
in terms of the optimal contract seen in the previous unit.

\section{2. A Simple Moral Hazard
Model}\label{a-simple-moral-hazard-model}

Let's assume that a company (principal) hires a manager (agent)
responsible for running the business. The company's outcomes depend on
both random factors (uncertainty) and the manager's effort. To simplify,
assume the effort can be either high (A) or low (B), and the company's
outcomes can either be success (E) or failure (F). The company's profit
in the success scenario is \$360, and in the failure scenario, it is
\$200. The probability of success is 75\% with high effort and 25\% with
low effort.

Therefore, the expected profit of the principal when effort is high is:

{[} 360 \times 0.75 + 200 \times 0.25 = 320 {]}

And when the effort is low:

{[} 360 \times 0.25 + 200 \times 0.75 = 240 {]}

This can be summarized in the following table:

\begin{longtable}[]{@{}lll@{}}
\toprule\noalign{}
& High Effort & Low Effort \\
\midrule\noalign{}
\endhead
\bottomrule\noalign{}
\endlastfoot
Success & 75\% & 25\% \\
Failure & 25\% & 75\% \\
Result & 320 & 240 \\
\end{longtable}

We assume that the principal seeks to maximize the expected value of
their net profits (expected value minus the payment to the agent). The
principal is assumed to be risk-neutral.

The agent, on the other hand, aims to maximize an expected utility
function of the form:

{[} U = U(w) - v(e) {]}

Where ( U(w) ) is the utility of the payment received (remuneration),
and ( v(e) ) is the cost of effort (e).

We can distinguish four possible cases:

\begin{enumerate}
\def\labelenumi{\arabic{enumi}.}
\tightlist
\item
  Observable effort, risk-neutral agent
\item
  Non-observable effort, risk-neutral agent
\item
  Observable effort, risk-averse agent
\item
  Non-observable effort, risk-averse agent
\end{enumerate}

\subsection{2.1 Observable Effort, Risk-Neutral
Agent}\label{observable-effort-risk-neutral-agent}

Let's assume an agent with a utility function ( U = w - v(e) ), and a
reservation utility of ( U = 81 ). The costs of effort are ( v(B) = 0 )
and ( v(A) = 63 ). (The numbers are arbitrary and chosen to facilitate
later comparisons).

If the principal settles for low effort, they offer the agent ( w = 81
). The expected profit for the principal is 240, and the net benefit is
( 240 - 81 = 159 ).

If the principal wants to ensure high effort, the agent's remuneration
must cover the disutility of effort, i.e., ( 81 + 63 = 144 ). The net
profit for the principal in this case is ( 320 - 144 = 176 ).

Thus, for the principal, the best alternative is to offer a fixed
payment of 144 and obtain an expected profit of 176.

\subsection{2.2 Non-Observable Effort, Risk-Neutral
Agent}\label{non-observable-effort-risk-neutral-agent}

Now, consider the case where the effort is non-observable but the agent
is risk-neutral. In this case, the conflict between incentives and risk
distribution is easily resolved. Since the agent is risk-neutral, the
absolute priority is on the incentive system. The most effective way to
provide incentives is to make the agent bear all the consequences of
their decisions.

The contract cannot depend on effort because effort is neither
observable nor verifiable. However, results such as profits are
observable. Therefore, the principal can exploit the correlation between
effort and profits to incentivize the agent. The principal will pay ( X
) in case of success (E), and ( Y ) in case of failure (F). This
introduces uncertainty for the agent.

The principal can induce high effort by offering the agent a contract
characterized by ( W\_E ) and ( W\_F ), with the following requirements:

\begin{enumerate}
\def\labelenumi{\arabic{enumi}.}
\tightlist
\item
  \textbf{Individual Rationality}: The utility for the agent must be
  greater than their reservation utility.
\item
  \textbf{Incentive Compatibility}: The utility of exerting high effort
  must be greater than the utility of low effort, i.e., ( EU(A)
  \textgreater{} EU(B) ).
\end{enumerate}

In our example, the constraints can be expressed as:

{[} 0.75 W\_E + 0.25 W\_F - 63 \textgreater{} 81 {]} {[} 0.75 W\_E +
0.25 W\_F - 63 \textgreater{} 0.25 W\_E + 0.75 W\_F {]}

The optimal contract in this case involves:

{[} W\_E = 184, \quad W\_F = 24 {]} \#\# 2.3 Observable Effort,
Risk-Averse Agent

Now, let's examine what happens when we have a risk-averse agent. For
example, assume the utility function ( U = W\^{}\{1/2\} - v(e) ), where
( v(A) = 3 ) and ( v(B) = 0 ), and the reservation utility is ( U = 9 ).

To induce low effort, it is sufficient to offer a fixed salary that
yields a utility level equal to or greater than the reservation utility.
The principal's expected net profit is ( 240 - 81 = 159 ).

To induce high effort, the agent needs to be compensated for the
additional disutility of the high effort. This can be achieved with a
fixed salary of 144, conditioned on high effort (since in this case,
effort can be observed). Thus, the expected net profit for the principal
is ( 320 - 144 = 176 ), higher than the profit for low effort (though it
may not always be optimal to pay for high effort).

In this case, the agent's risk aversion does not pose a problem since
there is no uncertainty involved. The agent's action and payment are not
contingent on the state of nature. All the uncertainty falls on the
principal. Additionally, if the agent has bargaining power, they could
negotiate up to 161 for high effort (we will explore this through
Edgeworth box representation).

\subsection{2.4 Non-Observable Effort and Risk-Averse
Agent}\label{non-observable-effort-and-risk-averse-agent}

The lack of observability of effort introduces the alternative of making
remuneration contingent on outcomes, which are correlated with effort,
or abandoning variable compensation altogether. This is where the main
conflict arises:

\begin{itemize}
\item
  \textbf{Efficient Risk Distribution}: As discussed earlier, the
  neutral party (the principal) should bear all the risk, while the
  risk-averse party (the agent) remains on the certainty line.
\item
  \textbf{Incentive Problem}: For proper incentives, the agent must
  perceive differences in remuneration based on their effort level.
\end{itemize}

The challenge is to find the optimal balance between incentive provision
and risk distribution. In the case of a risk-neutral agent, this was not
a significant issue, as simply transferring residual control to the
agent solved the problem. However, in this case, any variable
compensation scheme must compensate the agent for the risk they assume.

We retain the utility function and parameters from the previous section.

If the principal desires low effort, a fixed salary of 81 is sufficient.

However, to incentivize high effort, the contract with ( w = 144 ) will
no longer suffice because effort is not observable, and uniform wages
provide an incentive to cheat by exerting low effort.

The contract that induces high effort must meet both the individual
rationality and incentive compatibility constraints, expressed as:

{[} 0.75U(W\_E) + 0.25U(W\_F) - 3 \textgreater{} 9 {]} {[} 0.75U(W\_E) +
0.25U(W\_F) - 3 \textgreater{} 0.25U(W\_E) + 0.75U(W\_F) {]}

The second constraint can be written as:

{[} 0.5{[}U(W\_E) - U(W\_F){]} \textgreater{} 3 {]}

This captures the difference in utilities associated with the
remuneration in each state, ensuring that high effort is chosen over low
effort.

It is useful to express remuneration in terms of utility, such that (
X\_E = U(W\_E) ) and ( X\_F = U(W\_F) ).

The constraints are now written as:

{[} 0.75 \times X\_E + 0.25 \times X\_F = 12 {]} {[} X\_E - X\_F = 6 {]}

From this, we obtain ( X\_E = 13.5 ) and ( X\_F = 7.5 ), so that ( W\_E
= 182.25 ) and ( W\_F = 56.25 ). The expected cost of the contract for
the principal is ( 0.75 \times 182.25 + 0.25 \times 56.25 = 159.75 ),
leaving them with an expected net benefit of ( 320 - 159.75 = 169.25 ).

It is important to note that inducing high effort now has a higher cost
for the principal compared to the observable effort case. The cost has
increased from 144 to 159.75. The additional 6.75 is the cost of
non-observability, which economically corresponds to compensating the
agent for the risk assumed.

In this case, the contract can be formulated as a fixed payment plus a
share in the profits ( W = f + s\pi ).

\subsection{2.5 General Model}\label{general-model}

The moral hazard model presented in the previous section is a simplified
version based on the work of Holmstrom and others. In this section, we
present a more general model. It is important to note that the moral
hazard problem can be modeled as a sequential game where, in the first
stage, the principal offers a contract, and in the second stage, the
agent must accept or reject it and then choose their level of effort.
Thus, it is a problem that can be solved using backward induction.

In the final stage of the game, the agent chooses a level of effort:

{[} e \in \max \sum\_\{i=1\}\^{}n p\_i(e)U(w(x\_i)) - v(e) {]}

This condition corresponds to the incentive compatibility constraint. In
the immediately preceding stage, the agent decides whether to
participate in the contract:

{[} \sum\_\{i=1\}\^{}n p\_i(e)U(w(x\_i)) - v(e) \geq U {]}

This corresponds to the participation constraint.

In the first stage of the problem, the principal designs the contract,
anticipating the agent's behavior. Formally, the contract proposed by
the principal is the solution to:

{[} \max \sum\_\{i=1\}\^{}n p\_i(e){[}B(x\_i - w(x\_i)){]} {]}

Subject to:

{[} U \leq \sum\emph{\{i=1\}\^{}n p\_i(e)U(w(x\_i)) - v(e) {]} {[} e
\in \max \sum}\{i=1\}\^{}n p\_i(e)U(w(x\_i)) - v(e) {]} \#\# 2.5.1 Agent
Chooses Between Two Levels of Effort

Initially, we assume that the agent's effort can take two values: High
and Low (( e\_H ) and ( e\_L )), with ( e\_H \textgreater{} e\_L ) and (
v(e\_H) \textgreater{} v(e\_L) ). We denote ( p\_H\^{}i ) as the
probability distribution of results when effort is high, and ( p\_L\^{}i
) as the probability distribution of results when effort is low.

We assume that:

{[} \sum\emph{\{i=1\}\^{}k p\_H\^{}i \geq \sum}\{i=1\}\^{}k p\_L\^{}i
\quad \text{for all} \quad k = 1, 2, \dots, n - 1 {]}

{[} \sum\emph{\{i=1\}\^{}n p\_H\^{}i = \sum}\{i=1\}\^{}n p\_L\^{}i = 1
{]}

This implies that the probability distribution with high effort
stochastically dominates the distribution with low effort. In simpler
terms, poor results are more likely when the agent exerts low effort.

Suppose the principal wants to induce high effort from the agent. The
principal's problem is:

{[} \max \sum\_\{i=1\}\^{}n p\_i(e)(x\_i - w(x\_i)) {]}

Subject to:

{[} U \leq \sum\_\{i=1\}\^{}n p\_i(e)U(w(x\_i)) - v(e) {]}

{[} \sum\emph{\{i=1\}\^{}n p\_H\^{}i U(w(x\_i)) - v(e\_H)
\geq \sum}\{i=1\}\^{}n p\_H\^{}i U(w(x\_i)) - v(e\_L) {]}

(We assume that the principal is risk-neutral.)

The Lagrangian for the principal's problem is:

{[} L = \sum\_\{i=1\}\^{}n p\_i(e)(x\_i - w(x\_i)) +
\lambda \left[\sum_{i=1}^n p_H^i U(w(x_i)) - v(e_H) - U \right] +
\mu \left[ \sum_{i=1}^n (p_H^i - p_L^i) U(w(x_i)) - v(e_H) + v(e_L) \right]{]}

The first-order condition for the principal's problem is:

{[} \frac{\partial L}{\partial w(x_i)} = -p\_H\^{}i + \lambda p\_H\^{}i
U'(w(x\_i)) + \mu (p\_H\^{}i - p\_L\^{}i) U'(w(x\_i)) = 0 {]}

This simplifies to:

{[} p\_H\^{}i U'(w(x\_i)) = \lambda p\_H\^{}i + \mu (p\_H\^{}i -
p\_L\^{}i) {]}

Summing for ( i = 1, n ), and recalling that ( \sum\emph{\{i=1\}\^{}n
p\_H\^{}i = \sum}\{i=1\}\^{}n p\_L\^{}i = 1 ), we have:

{[} \sum\_\{i=1\}\^{}n p\_H\^{}i U'(w(x\_i)) = \lambda \textgreater{} 0
{]}

Since ( \lambda \textgreater{} 0 ), the participation constraint is
binding. Rewriting the first-order condition, we get:

{[} \frac{1}{U'(w(x_i))} = \lambda +
\mu \left[ 1 - \frac{p_L^i}{p_H^i} \right]{]}

\subsubsection{Implications:}\label{implications}

\begin{itemize}
\tightlist
\item
  If ( \mu = 0 ), the salary is constant, which means that the incentive
  compatibility condition is not satisfied.
\item
  ( \mu \textgreater{} 0 ) implies that the shadow price of the
  constraint is positive, meaning that the principal faces a cost
  greater than zero.
\item
  ( \mu \textgreater{} 0 ) also means that the salary varies according
  to the outcome, with higher salaries for outcomes where (
  \frac{p_L^i}{p_H^i} ) is smaller.
\end{itemize}

The ratio ( \frac{p_L^i}{p_H^i} ) is known as the likelihood ratio and
indicates how strongly a particular result suggests that high effort was
exerted. The necessary condition for a better result to be associated
with a higher salary is that ( \frac{p_L^i}{p_H^i} ) must be decreasing
in ( i ), a property known as the monotonous likelihood quotient.

In general, to fully characterize the optimal contract under moral
hazard, even in simple cases, we need to impose additional structure on
the problem, which comes at the cost of generality.

\subsection{2.5.2 Effort as a Continuous Variable: First-Order
Approach}\label{effort-as-a-continuous-variable-first-order-approach}

Solving the problem when effort is a continuous variable (e.g., ( e
\in [0, 1] )) is challenging because it involves a double maximization,
with one of the constraints being a second maximization problem. To
resolve this difficulty, Holmstrom (1979) proposed replacing the agent's
maximization problem (incentive compatibility constraint) with its
first-order condition:

{[} \sum\_\{i=1\}\^{}n p'\_i(e) U(w(x\_i)) - v'(e) = 0 {]}

The issue with this method is that the first-order condition is
necessary but not sufficient. Effort is present in both the disutility
function ( v(e) ) and the probability distribution of results, implying
that the expected utility function is not necessarily concave.
Therefore, the first-order approach may yield more solutions than the
original problem, potentially leading to suboptimal outcomes for the
principal.

Assuming the first-order condition is both necessary and sufficient, the
first-order approach is valid, and the principal's maximization problem
becomes:

{[} \max \sum\_\{i=1\}\^{}n p\_i(e)(x\_i - w(x\_i)) {]}

Subject to:

{[} U \leq \sum\_\{i=1\}\^{}n p\_i(e) U(w(x\_i)) - v(e) {]}

{[} \sum\_\{i=1\}\^{}n p'\_i(e) U(w(x\_i)) - v'(e) = 0 {]}

The Lagrangian for this problem is:

{[} L = \sum\_\{i=1\}\^{}n p\_i(e)(x\_i - w(x\_i)) +
\lambda \left[\sum_{i=1}^n p_H^i U(w(x_i)) - v(e_H) - U \right] +
\mu \left[ \sum_{i=1}^n p'_i(e) U(w(x_i)) - v'(e) \right]{]}

The first-order condition for the principal's problem is:

{[} \frac{\partial L}{\partial w(x_i)} = -p'\_i(e) + \lambda p\_i(e)
U'(w(x\_i)) + \mu p'\_i(e) U'(w(x\_i)) = 0 {]}

Rewriting this, we have:

{[} \frac{1}{U'(w(x_i))} = \lambda + \mu \frac{p'_i(e)}{p_i(e)} {]}

\subsubsection{Implications:}\label{implications-1}

If ( \frac{p'_i(e)}{p_i(e)} ) is increasing in ( i ), then the salary
also increases with ( i ). The condition that the likelihood quotient is
increasing implies that a better result indicates a higher probability
of high effort. This result is similar to the case with two effort
levels.

Since the first-order approach does not always lead to valid solutions,
we must impose restrictions on the conditional distribution of outcomes
given effort to ensure that the second-order condition holds. One
example of this approach is provided by Hart and Holmstrom (1987), which
we will examine next.

\subsection{2.5.3 Linearity in the Distribution
Function}\label{linearity-in-the-distribution-function}

Suppose effort can take any value in the interval ( {[}0, 1{]} ), and
the probability distribution of outcomes is given by:

{[} p\_i(e) = e p\_H\^{}i + (1 - e) p\_L\^{}i {]}

Hart and Holmstrom refer to this as linearity in the distribution
function. This can be interpreted as an extension of the case where
effort could take only two values, allowing the agent to choose a mixed
strategy.

The agent's expected utility as a function of effort (incentive
compatibility condition) is:

{[} E{[}U(e){]} = \sum\_\{i=1\}\^{}n
\left[ e p_H^i + (1 - e) p_L^i \right] U(w(x\_i)) - v(e) {]}

This can be rewritten as:

{[} E{[}U(e){]} = \sum\emph{\{i=1\}\^{}n p\_L\^{}i U(w(x\_i)) + e
\sum}\{i=1\}\^{}n (p\_H\^{}i - p\_L\^{}i) U(w(x\_i)) - v(e) {]}

It can be shown that in this case, the expected utility function is
concave in effort.

The first-order condition for maximizing the agent's utility is:

{[} \sum\_\{i=1\}\^{}n (p\_H\^{}i - p\_L\^{}i) U(w(x\_i)) - v'(e) = 0
{]}

The principal's maximization problem is:

{[} \max \sum\_\{i=1\}\^{}n \left[ e p_H^i + (1 - e) p_L^i \right] (x\_i
- w(x\_i)) {]}

Subject to:

{[} U \leq \sum\_\{i=1\}\^{}n
\left[ e p_H^i + (1 - e) p_L^i \right] U(w(x\_i)) - v(e) {]}

{[} \sum\_\{i=1\}\^{}n (p\_H\^{}i - p\_L\^{}i) U(w(x\_i)) - v'(e) = 0
{]}

The Lagrangian for this problem is:

{[} L = \sum\_\{i=1\}\^{}n \left[ e p_H^i + (1 - e) p_L^i \right] (x\_i
- w(x\_i)) +
\lambda \left[ \sum_{i=1}^n \left[ e p_H^i + (1 - e) p_L^i \right] U(w(x\_i))
- v(e) - U \right{]} +
\mu \left[ \sum_{i=1}^n (p_H^i - p_L^i) U(w(x_i)) - v'(e) \right]{]}

The first-order condition for the principal's problem is:

{[} \frac{\partial L}{\partial w(x_i)} = - \left( e p\_H\^{}i + (1 - e)
p\_L\^{}i \right) + \lambda \left( e p\_H\^{}i + (1 - e) p\_L\^{}i
\right) U'(w(x\_i)) + \mu (p\_H\^{}i - p\_L\^{}i) U'(w(x\_i)) = 0 {]}

This reduces to:

{[} \frac{1}{U'(w(x_i))} = \lambda +
\mu \frac{(p_H^i - p_L^i)}{e p_H^i + (1 - e) p_L^i} {]}



\end{document}
